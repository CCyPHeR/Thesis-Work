\chapter*{Abstract}
%\setheader{Abstract}

Buoyancy-driven bubbly flow widely exists in equipment like bubble columns, electrochemical cells, etc. The influence of bubbles on the overall velocity and local flow behaviours is complex and still under debate these days. 

In this study, the focus is put on the numerical modelling of buoyancy-driven flow in a circulating channel. The circulating channel is composed of a riser channel and a downcomer. The gas is evolved from the vertical wall along the riser channel, forming a bubble plume covering the wall. In the context of electrochemistry, this kind of setup is commonly used for electrolysers, and the existence of the bubble plume could result in a decrease in the reaction surface and overall efficiency. Therefore, studying the behaviour of the flow inside the circulating channel and relation between certain operating parameters is important.

First, a multiphase flow model is used to simulate the buoyancy-driven flow in a vertical channel without the overall circulation. The purpose is to reproduce the results of experiments and simulations from literature to prove the validity of the numerical model used in this work, for lack of detailed experimental data for electro-generated bubbles in the circulating channel. The velocity profile across the channel matches well with the experimental data under the same operating conditions, while the volume fraction qualitatively matches the simulation from other works.

Afterward, the same multiphase model is used to simulate the buoyancy-driven circulating bubbly flow. A parametric study is conducted to investigate the variation of the volume fraction profile, the overall circulating velocity, and local velocity profile under different current density and riser channel width. 

It is found that when the channel width is small ( $< 0.5 \mathrm{cm}$), the plume thickness decreases with an increasing channel width and a decreasing current density. However, as the channel width keeps increasing, the bubble plume thickness gradually becomes independent of current density and channel width. 

Regarding the velocity, as the riser channel gap increases, the circulating velocity first increases then decreases, reaching its maximum value at around $0.2 \, \mathrm{cm}$ channel width. Based on the velocity profile from the simulation, it is because the velocity profile gradually deviates from the solution of Poiseuille flow. Analytical solutions based on certain assumptions are given to estimate the circulating velocity.